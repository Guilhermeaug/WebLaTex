\documentclass[12pt,a4paper]{article}
\usepackage{times}
\usepackage[a4paper, margin=1in]{geometry}
\usepackage[utf8]{inputenc}
\usepackage[T1]{fontenc}
\usepackage[portuguese]{babel}
\usepackage{amsmath}
\usepackage{graphicx}

\title{Regressão Linear com uma Variável}
\author{Guilherme Augusto, Yasmim Augusta}
\date{\today}

\begin{document}

\maketitle

1. O que é chamada de regressão logística?\\

A regressão logística é um modelo estatístico utilizado para prever a probabilidade de ocorrência de um evento binário, com base em uma ou mais variáveis independentes. Ela é amplamente utilizada em problemas de classificação, nos quais o objetivo é prever a classe de um objeto com base em suas características.

\vspace{1cm}

2. Quais são as técnicas utilizadas no treinamento semi-supervisionado?\\

No treinamento semi-supervisionado, são utilizadas técnicas que combinam dados rotulados e não rotulados para treinar um modelo de aprendizado de máquina. Exemplos:
\begin{itemize}
    \item Propagação de rótulos: Atribui rótulos aos dados não rotulados com base nos dados rotulados mais próximos.
    \item Co-training: Treina o modelo em duas visualizações diferentes dos dados, cada uma com um conjunto diferente de características.
    \item Aprendizado ativo: Seleciona os exemplos mais informativos dos dados não rotulados para serem rotulados pelo usuário.
    \item Treinamento por reforço: Utiliza recompensas para guiar o treinamento do modelo.
\end{itemize}

\vspace{1cm}

3. Quais são as técnicas utilizadas no treinamento por reforço?\\

No treinamento por reforço, são utilizadas técnicas que utilizam recompensas para guiar o treinamento do modelo. Um exemplo é o gradiente de política, que ajusta a política do agente com base no gradiente da recompensa esperada.

\vspace{1cm}

4. Explique a diferença entre Sum of Squared Error (SSE) e Total Sum of Squares (TSS).\\

O erro quadrático total (SSE) é a soma dos quadrados das diferenças entre os valores observados e os valores previstos pelo modelo. Ele é utilizado para avaliar o ajuste do modelo aos dados. Já a soma total dos quadrados (TSS) é a soma dos quadrados das diferenças entre os valores observados e a média dos valores observados. Ela é utilizada para avaliar a variabilidade dos dados em relação à média.

\end{document}
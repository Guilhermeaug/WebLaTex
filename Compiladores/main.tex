\documentclass[12pt,a4paper]{article}
\usepackage{times}
\usepackage[a4paper, margin=1in]{geometry}
\usepackage[utf8]{inputenc}
\usepackage[T1]{fontenc}
\usepackage[portuguese]{babel}
\usepackage{amsmath}
\usepackage{graphicx}

\title{Geração de Código}
\author{Guilherme Augusto de Oliveira}
\date{\today}

\begin{document}

\maketitle

\section{Representação Intermediária}

Após a análise semântica, muitos compiladores geram uma representação intermediária do código fonte para facilitar a geração posterior de código. Uma de suas vantagens é que esse tipo de representação é mais fácil de ser manipulada do que o código fonte original. Além disso, por ser mais próximo da linguagem de máquina, é possível otimizar o código gerado para diferentes arquiteturas.

Uma das formas de represenção intermediária é o código de três endereços. Nesse esquema, cada instrução é composta por, no máximo, três endereços. Por exemplo: $ x = y op z $, onde $ x $, $ y $ e $ z $ são endereços de memória e $ op $ é um operador. Os endereços podem ser representados por um nome símbolico (por conveniência) ou por um código gerado pelo compilador.

Um conceito importante que envolve o código intermediário são as instruções suportadas. As instruções de três endereços mais comuns são: atribuição, operações aritméticas, operações lógicas, operações de desvio condicional e operações de desvio incondicional. Todo o código fonte é traduzido para essas instruções, que são bem parecidas com a representação em código de máquina.

\subsection{Memória}

O tipo de dado e o tamanho de bytes necessários para armazenar um valor em memória são definidos na tabela de símbolos por meio de atributos semânticos. Desse modo, a partir do nome do identificador é possível a quantidade de memória necessária para armazenar o valor associado a ele.

A memória é alocada em sequência, ou seja, o endereço de memória de um identificador é calculado a partir do endereço do identificador anterior. Por exemplo, se o endereço do identificador $ x $ é $ 1000 $ e o tamanho de $ x $ é $ 4 $ bytes, então o endereço de $ y $ é $ 1004 $. Para manter esse controle, o esquema de tradução mantém uma varíavel interna conhecida como \textit{offset}.

\subsection{Processo de Tradução}

Para traduzir as expressões na forma de três endereços, são utilizados atributos semânticos e ações de tradução auxiliares.
\begin{itemize}
    \item addr: nome, constante ou identificador temporário que indica o endereço de memória de uma expressão.
    \item top: entrada correspondente ao topo da pilha de operandos.
    \item gen: gera uma instrução de três endereços.
    \item newtemp: é uma função auxiliar que retorna um novo identificador temporário.
\end{itemize}

Fluxos de controle envolvem o uso de rótulos para marcar o início e o fim de um bloco de código. O rótulo é um identificador único que indica o endereço de memória de uma instrução. O esquema de tradução mantém uma variável interna que armazena o rótulo atual. Outra forma de representar endereços é utilizar diretamente o valor.

O principal problema na geração de código é a presença de desvios condicionais. Para resolver esse problema, é utilizada a técnica de remendo (\textit{backpatching}). Essa técnica consiste em preencher os endereços de memória dos desvios condicionais após a geração de todo o código. Para isso, é necessário manter uma lista de endereços de memória que precisam ser preenchidos. Isso é feito utilizando os atributos \textit{truelist} e \textit{falselist}. O primeiro armazena os endereços de memória que precisam ser preenchidos com o endereço de memória da próxima instrução, caso a condição seja verdadeira. O segundo armazena os endereços de memória que precisam ser preenchidos com o endereço de memória da próxima instrução, caso a condição seja falsa.

Para a manutenção das listas são utilizadas as funções auxiliares \textit{makelist}, \textit{merge} e \textit{backpatch}. A função \textit{makelist} cria uma nova lista de endereços de memória. A função \textit{merge} junta duas listas de endereços de memória. A função \textit{backpatch} preenche os endereços de memória com o endereço de memória da próxima instrução.

Exemplo: $ if \; x \; relop \; y \; then \; S_1 \; else \; S_2 $. Nesse caso, a lista \textit{truelist} é
preenchida com o endereço de memória da instrução $ S_1 $ e a lista \textit{falselist} é preenchida com o endereço de memória da instrução $ S_2 $.

\section{Conclusão}

É possível perceber que a geração de código intermediário facilita a utilização do código fonte original para diferentes arquiteturas. Existem várias técnicas para isso, sendo a geração de código de três endereços uma das mais comuns.

A próxima etapa é a geração de código de máquina, que pode ser executada diretamente pelo processador. Para isso, o intermediário é re-escrito com referências de registradores, endereços de memória (se ainda não estiverem presentes) e instruções de máquina. Sendo que cada arquitetura possui suas particularidades de implementação. O lado bom é que é mais fácil otimizar o código final tendo a representação intermediária.

\end{document}